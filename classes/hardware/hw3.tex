\iffalse
// c precompiler stuff
#define cnote \
\begin{lstlisting}[ \
  mathescape, \
  columns=fullflexible, \
  basicstyle=\fontfamily{lmvtt}\selectfont, \
] 

#define endcnote \end{lstlisting}
\fi

\documentclass{article} 
\usepackage{amsthm} % writing proofs
\usepackage{amsfonts} % for blackboard bold charactes like Z R
\usepackage{amssymb} % came with intermediate value thrm
\usepackage[fleqn]{amsmath} % for arguments vertically underneath like lim, piecwise functions, equality arrays
\setlength{\mathindent}{0in} %neutralizes equation indentation
\usepackage{listings} % more literal, makes hw easier
\setlength{\parindent}{0in} %neutralizes annoying paragraph indent
\usepackage{mathtools} % for floor and ceiling functions / macros
\usepackage{nccmath} % for centering equations with ceqn
\usepackage{tkz-euclide} % geometry
\usetkzobj{all} % include all objects of tkz

\DeclarePairedDelimiter\ceil{\lceil}{\rceil}
\DeclarePairedDelimiter\floor{\lfloor}{\rfloor}

\newcommand{\f}{\frac}
\newcommand{\ra}{\rightarrow}
\newcommand{\rla}{\leftrightarrow}
\newcommand{\lra}{\leftrightarrow}
\newcommand{\bb}{\mathbb}

%\renewcommand{\qedsymbol}{$\blacksquare$}
\renewcommand{\qedsymbol}{$\dashv$}

\begin{document}
cnote

Jordan Winkler
computer architecture
Sun Jan 27 14:41:03 EST 2019
hw 3

1) Do Exercise 2.2 on page 164 of the textbook.

For the following MIPS assembly instructions above, what is a
corresponding C statement?
add f, g, h
add f, i, f

f = g + h;
f = i + f;

2) Write MIPS code to implement the following C/C++ command:
Spring 2019
j = i * 2;

# j = i * 2;
# s0 = j, a0 = i 

add \$s0, \$a0, \$a0


3) Do Exercise 2.3 on page 165 of the textbook.

For the following C statement, what is the corresponding
MIPS assembly code? Assume that the variables f , g , h , i , and j are assigned to
registers $s0 , $s1 , $s2 , $s3 , and $s4 , respectively. Assume that the base address
of the arrays A and B are in registers $s6 and $s7 , respectively.
B[8] = A[i−j];

4) Do Exercise 2.4 on page 165 of the textbook.

For the MIPS assembly instructions below, what is the
corresponding C statement? Assume that the variables f , g , h , i , and j are assigned
to registers $s0 , $s1 , $s2 , $s3 , and $s4 , respectively. Assume that the base address
of the arrays A and B are in registers $s6 and $s7 , respectively.




5) Assume that register $t0 holds the base address in memory of array ‘A’ and register $t1
holds the base address in memory of array ‘B’. Write MIPS code to swap the contents of
A[0] with B[0] and A[1] with B[1].



endcnote
\end{document}