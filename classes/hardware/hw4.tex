\iffalse
// c precompiler stuff
#define cnote \
\begin{lstlisting}[ \
  mathescape, \
  columns=fullflexible, \
  basicstyle=\fontfamily{lmvtt}\selectfont, \
] 

#define endcnote \end{lstlisting}
\fi

\documentclass{article} 
\usepackage{amsthm} % writing proofs
\usepackage{amsfonts} % for blackboard bold charactes like Z R
\usepackage{amssymb} % came with intermediate value thrm
\usepackage[fleqn]{amsmath} % for arguments vertically underneath like lim, piecwise functions, equality arrays
\setlength{\mathindent}{0in} %neutralizes equation indentation
\usepackage{listings} % more literal, makes hw easier
\setlength{\parindent}{0in} %neutralizes annoying paragraph indent
\usepackage{mathtools} % for floor and ceiling functions / macros
\usepackage{nccmath} % for centering equations with ceqn
\usepackage{tkz-euclide} % geometry
\usetkzobj{all} % include all objects of tkz

\DeclarePairedDelimiter\ceil{\lceil}{\rceil}
\DeclarePairedDelimiter\floor{\lfloor}{\rfloor}

\newcommand{\f}{\frac}
\newcommand{\ra}{\rightarrow}
\newcommand{\rla}{\leftrightarrow}
\newcommand{\lra}{\leftrightarrow}
\newcommand{\bb}{\mathbb}

%\renewcommand{\qedsymbol}{$\blacksquare$}
\renewcommand{\qedsymbol}{$\dashv$}

\begin{document}

Jordan Winkler

Computer Architecture

Mon Feb  4 23:06:09 EST 2019

hw4 \\

1. Convert the hexadecimal number C 7 B 4 9 D to binary. \\

$(12, 7, 11, 4, 9, 13)_{16} = (1100 0111 1011 0100 1001 1101)_2$ \\

2. Convert the decimal number 315 directly into octal - do NOT convert to binary, hex, etc. as an intermediate step. \\

315/8 = 39 R 3

39/8 = 4 R 7

4/8 = 0 R 4 \\

$(473)_8$ \\

3. Using expanded notation, convert the hexadecimal number E 6 F 3 into decimal. \\

$(E6F3)_{16} = (14, 6, 15, 3)_{16} = (14 * 16^3 + 6 * 16^2 + 15 * 16^1 + 3 * 16^0)_{10} = 59123_{10}$ \\

4. Convert the decimal number -85 to the 2’s complement 8-bit form. \\

85/2 = 42 R 1

42/2 = 21 R 0

21/2 = 10 R 0

10/2 =  5 R 0

 5/2 =  2 R 1

 2/2 =  1 R 0

 1/2 =  0 R 1 \\
 

$01010101_2$ \\

$10101010_{1-complement}$ \\

$10101011_{2s-complent}$ \\

5. If I have 198 unique items to represent, how many bits do I need to do this? \\

$\ceil{log_2(198)} = 8_{10} $ bits\\

6. Consider the number: 1 0 1 1 0 0 1 1 \\

a) What is this, in decimal, if it is an unsigned binary number? \\

$(1 0 1 1 0 0 1 1)_2 = (128 + 32 + 16 + 2 + 1)_{10} = 179_{10}$ \\

b) What is this, in decimal, if it is a ones complement number? \\

$(1 0 1 1 0 0 1 1)_{1-comp} = -(1001100)_2 = -(4 + 8 + 64)_{10} = -76_{10}$ \\

c) What is this, in decimal, if it is a twos complement number? \\

$(1 0 1 1 0 0 1 1)_{2-comp} = -(1001101)_2 = -(1 + 4 + 8 + 64)_{10} = -77_{10} $ \\ 

7. Convert the binary number 0 1 1 0 0 1 1 0 1 0 1 0 1 into octal. \\

$(0 110 011 010 101)_2 = (110 011 010 101)_2 = (6325)_8$ \\

8. Consider the number: 0 1 0 1 1 0 0 1 \\

a) What is this, in decimal, if it is an unsigned binary number? \\

$(01011001)_2 = (1 + 8 + 16 + 32 + 128)_{10} = (185)_{10}$ \\

b) What is this, in decimal, if it is a ones complement number? \\

$(01011001)_{1-comp} = (1 + 8 + 16 + 32 + 128)_{10} = (185)_{10}$ \\

c) What is this, in decimal, if it is a twos complement number? \\

$(01011001)_{2-comp} = (1 + 8 + 16 + 32 + 128)_{10} = (185)_{10}$ \\


d) What is this, in decimal, if it is a sign-magnitude number? \\

$(01011001)_{sign-mag} = (1 + 8 + 16 + 32 + 128)_{10} = (185)_{10}$

\end{document}