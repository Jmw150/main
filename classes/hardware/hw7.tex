\iffalse
// c precompiler stuff
#define cnote \
\begin{lstlisting}[ \
  mathescape, \
  columns=fullflexible, \
  basicstyle=\fontfamily{lmvtt}\selectfont, \
] 

#define endcnote \end{lstlisting}
\fi

\documentclass{article} 
\usepackage{amsthm} % writing proofs
\usepackage{amsfonts} % for blackboard bold charactes like Z R
\usepackage{amssymb} % came with intermediate value thrm
\usepackage[fleqn]{amsmath} % for arguments vertically underneath like lim, piecwise functions, equality arrays
\setlength{\mathindent}{0in} %neutralizes equation indentation
\usepackage{listings} % more literal, makes hw easier
\setlength{\parindent}{0in} %neutralizes annoying paragraph indent
\usepackage{mathtools} % for floor and ceiling functions / macros
\usepackage{nccmath} % for centering equations with ceqn
\usepackage{tkz-euclide} % geometry
\usetkzobj{all} % include all objects of tkz

\DeclarePairedDelimiter\ceil{\lceil}{\rceil}
\DeclarePairedDelimiter\floor{\lfloor}{\rfloor}

\newcommand{\f}{\frac}
\newcommand{\ra}{\rightarrow}
\newcommand{\rla}{\leftrightarrow}
\newcommand{\lra}{\leftrightarrow}
\newcommand{\bb}{\mathbb}

%\renewcommand{\qedsymbol}{$\blacksquare$}
\renewcommand{\qedsymbol}{$\dashv$}

\begin{document}

Jordan Winkler

Computer Architecture

Mon Apr  1 23:43:45 EDT 2019

hw7\\

1) Do Exercise B.2 on page B-80 of the textbook. \\

Prove that the two equations for E in the example starting on page B-7 are equivalent by using DeMorgan's theorems and the axioms shown on page B-7. \\

$E_1 = (A B + A C + B C)(A B C)'$

$E_2 = A B C' + A C B' + B C A'$

\begin{eqnarray*}
E_1 & = & (A B + A C + B C)(A B C)' \\
    & = & (A B + A C + B C)(A'+ B'+ C') \\
    & = & ABA' + ACA' + BCA' + ABB' + ACB' + BCB' + ABC' + ACC' + BCC' \\
    & = & AA'B + AA'C + BCA' + ABB' + ACB' + BB'C + ABC' + ACC' + BCC' \\
    & = & ACB' + ABC' + BCA' \\
    & = & E_2 
\end{eqnarray*}

2) Do Exercise B.5 on page B-80 of the textbook. \\

Prove that the NOR gate is universal by showing how to build
the AND, OR, and NOT functions using a two-input NOR gate. \\

$A' = (A + A)'$ \\

$AB = (A' + B')' = ((A+A)' + (B+B)')'$ \\

$A+B = (A'B')' = ((A+B)')' = ((A+B)' + (A+B)')'$ \\

3) User perfect induction to prove or disprove $(A) (A' + B) = AB$ \\

\begin{verbatim}
+-------+-------+-------+-------+-------+
|   A   |   B   |  A'+B |A(A'+B)|  AB   |
+-------+-------+-------+-------+-------+
|   0   |   0   |   1   |   0   |   0   |
+-------+-------+-------+-------+-------+
|   0   |   1   |   1   |   0   |   0   |
+-------+-------+-------+-------+-------+
|   1   |   0   |   0   |   0   |   0   |
+-------+-------+-------+-------+-------+
|   1   |   1   |   1   |   1   |   1   |
+-------+-------+-------+-------+-------+
\end{verbatim}

So $(A) (A' + B) <-> AB$ is always 1, or in other words is a tautology. \\


4) Draw the truth table and the logic circuit for the following function $F = (A + B) \cdot (A' + C')$ \\

(Note, for the logic circuit part, you could draw it by hand) For the truth table, have Separate columns for ALL intermediate steps. \\

\begin{verbatim}
+----+----+----+----------+------------+------------+------------+------------+
| A  | B  | C  |    A'    |    C'      |   A+B      |  A'+C'     |(A+B)(A'+C')|
+----+----+----+----------+------------+------------+------------+------------+
| 0  | 0  | 0  |    1     |    1       |    0       |    1       |     0      |
+----+----+----+----------+------------+------------+------------+------------+
| 0  | 0  | 1  |    1     |    0       |    0       |    1       |     0      |
+----+----+----+----------+------------+------------+------------+------------+
| 0  | 1  | 0  |    1     |    1       |    1       |    1       |     1      |
+----+----+----+----------+------------+------------+------------+------------+
| 0  | 1  | 1  |    1     |    0       |    1       |    1       |     1      |
+----+----+----+----------+------------+------------+------------+------------+
| 1  | 0  | 0  |    0     |    1       |    1       |    1       |     1      |
+----+----+----+----------+------------+------------+------------+------------+
| 1  | 0  | 1  |    0     |    0       |    1       |    0       |     0      |
+----+----+----+----------+------------+------------+------------+------------+
| 1  | 1  | 0  |    0     |    1       |    1       |    1       |     1      |
+----+----+----+----------+------------+------------+------------+------------+
| 1  | 1  | 1  |    0     |    0       |    1       |    0       |     0      |
+----+----+----+----------+------------+------------+------------+------------+
\end{verbatim}

In scheme 
\begin{verbatim}
(define (F A B C) 
    (and (or A B) 
         (or (not A) (not C)))) 
\end{verbatim}

\vspace{5cm}


5) Do Exercise B.11 on page B-81 of the textbook. \\

Assume that X consists of 3 bits, x2 x1 x0. Write four logic functions that are true if and only if

X contains only one 0 \\

$x_1'x_2x_3 + x_1x_2'x_3 + x_1x_2x_3'$ \\

X contains an even number of 0s \\

$x_1'x_2'x_3 + x_1'x_2x_3' + x_1x_2'x_3'$ \\

X when interpreted as an unsigned binary number is less than 4 \\

$x_1'$ \\

X when interpreted as a signed (two's complement) number is negative \\

$x_1$ \\

And to double check \\

\begin{verbatim}
+----------+----------+----------+----------+----------+----------+----------+
|   x1     |   x2     |   x3     |  1 zero  | 2 zeroes |   0xx    |  1xx     |
+----------+----------+----------+----------+----------+----------+----------+
|    0     |    0     |    0     |    0     |    0     |    1     |   0      |
+----------+----------+----------+----------+----------+----------+----------+
|    0     |    0     |    1     |    0     |    1     |    1     |   0      |
+----------+----------+----------+----------+----------+----------+----------+
|    0     |    1     |    0     |    0     |    1     |    1     |   0      |
+----------+----------+----------+----------+----------+----------+----------+
|    0     |    1     |    1     |    1     |    0     |    1     |   0      |
+----------+----------+----------+----------+----------+----------+----------+
|    1     |    0     |    0     |    0     |    1     |    0     |   1      |
+----------+----------+----------+----------+----------+----------+----------+
|    1     |    0     |    1     |    1     |    0     |    0     |   1      |
+----------+----------+----------+----------+----------+----------+----------+
|    1     |    1     |    0     |    1     |    0     |    0     |   1      |
+----------+----------+----------+----------+----------+----------+----------+
|    1     |    1     |    1     |    0     |    0     |    0     |   1      |
+----------+----------+----------+----------+----------+----------+----------+
\end{verbatim}


6) Do Exercise B.14 on page B-81 of the textbook. \\

Implement a switching network that has two data inputs (A and B), two data outputs (C and D), and a control input (S). 
If S equals 1, the network is in pass-through mode, and C should equal A, and D should equal B. 
If S equals 0, the network is in crossing mode, and C should equal B, and D should equal A.




\end{document}