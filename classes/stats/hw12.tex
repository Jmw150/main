\iffalse
// c precompiler stuff
#define cnote \
\begin{lstlisting}[ \
  mathescape, \
  columns=fullflexible, \
  basicstyle=\fontfamily{lmvtt}\selectfont, \
] 

#define endcnote \end{lstlisting}
\fi

\documentclass{article} 
\usepackage{amsthm} % writing proofs
\usepackage{amsfonts} % for blackboard bold charactes like Z R
\usepackage{amssymb} % came with intermediate value thrm
\usepackage[fleqn]{amsmath} % for arguments vertically underneath like lim, piecwise functions, equality arrays
\setlength{\mathindent}{0in} %neutralizes equation indentation
\usepackage{listings} % more literal, makes hw easier
\setlength{\parindent}{0in} %neutralizes annoying paragraph indent
\usepackage{mathtools} % for floor and ceiling functions / macros
\usepackage{nccmath} % for centering equations with ceqn
%\usepackage{tkz-euclide} % geometry
%\usetkzobj{all} % include all objects of tkz

\DeclarePairedDelimiter\ceil{\lceil}{\rceil}
\DeclarePairedDelimiter\floor{\lfloor}{\rfloor}

\newcommand{\f}{\frac}
\newcommand{\ra}{\rightarrow}
\newcommand{\rla}{\leftrightarrow}
\newcommand{\lra}{\leftrightarrow}
\newcommand{\bb}{\mathbb}

%\renewcommand{\qedsymbol}{$\blacksquare$}
\renewcommand{\qedsymbol}{$\dashv$}

\begin{document}

Jordan Winkler

stats

Sun Apr 21 16:16:46 EDT 2019

Homework 12 
Section 10.8 Exercise 10.55, 10.60, 10.63. 10.64 \\

10.55 A marketing expert for a pasta-making company believes that 40\% of pasta lovers prefer lasagna. If 9 out of 20 pasta lovers choose lasagna over other pastas, what can be concluded about the expert’s claim? Use a 0.05 level of significance. \\

$H_0 : p \leq 0.4$

$H_a : p > 0.4$ \\

$n = 20$

$x = 9$

$\alpha = 0.05$

$E(X) = np = 8$ \\

$P(X > 8 | p = 0.4) = \sum^{20}_{x=9} choose(20,x)*.4^x*.6^{20-x} = 0.4044 > 0.05$ \\

Do not reject the null. \\


10.60 At a certain college, it is estimated that at most 25\% of the students ride bicycles to class. Does this seem to be a valid estimate if, in a random sample of 90 college students, 28 are found to ride bicycles to class? Use a 0.05 level of significance. \\

$H_0 : p \leq 0.25 $

$H_a : p > 0.25 $ \\

$\alpha = 0.05$

$p = 0.25$ 

$x = 28$

$n = 90$

$\hat{p} = x/n = 0.3111$ 

$Z = (\hat{p} - p)/(p(1-p)/n)^{1/2} = 1.3387$ \\

$P(Z > 1.3387) = 0.0901 > 0.05$ \\

Do not reject null. \\

10.63 In a study to estimate the proportion of residents in a certain city and its suburbs who favor the construction of a nuclear power plant, it is found that 63 of 100 urban residents favor the construction while only 59 of 125 suburban residents are in favor. Is there a significant difference between the proportions of urban and suburban residents who favor construction of the nuclear plant? Make use of a P-value. \\

$H_0 : p_1 = p_2$

$H_a : p_1 \neq p_2$ \\

$\hat{p_1} = x_1/n_1 = 63/100 = 0.63$

$\hat{p_2} = x_2/n_2 = 59/125 = 0.472$

$\hat{p} = (x_1 + x_2)/(n_1 + n_2) = $$0.542$

$z = (\hat{p_1} - \hat{p_2})/(\hat{p}(1 - \hat{p})(1/n_1 + 1/n_2))^{1/2}$$=2.393$ \\

$2P(Z < -2.393) = 0.0168 < 0.05$  \\

Reject null. \\


10.64 In a study on the fertility of married women conducted by Martin O’Connell and Carolyn C. Rogers for the Census Bureau in 1979, two groups of childless wives aged 25 to 29 were selected at random, and each was asked if she eventually planned to have a child. One group was selected from among wives married less than two years and the other from among wives married five years. Suppose that 240 of the 300 wives married less than two years planned to have children some day compared to 288 of the 400 wives married five years. Can we conclude that the proportion of wives married less than two years who planned to have children is significantly higher than the proportion of wives married five years? Make use of a P -value. \\

$H_0 : p_1 = p_2$

$H_1 : p_1 > p_2$ \\

$\hat{p_1} = x_1/n_1 = 240/300 = 0.8$

$\hat{p_2} = x_2/n_2 = 288/400 = 0.72$

$\hat{p} = (x_1 + x_2)/(n_1 + n_2) = $$0.75$

$z = (\hat{p_1} - \hat{p_2})/(\hat{p}(1 - \hat{p})(1/n_1 + 1/n_2))^{1/2}$$=2.433$ \\

$2P(Z > 2.433) = 0.0075 < 0.05$  \\

Reject null. \\


\end{document}
