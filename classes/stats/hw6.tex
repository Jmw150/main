\iffalse
// c precompiler stuff
#define cnote \
\begin{lstlisting}[ \
  mathescape, \
  columns=fullflexible, \
  basicstyle=\fontfamily{lmvtt}\selectfont, \
] 

#define endcnote \end{lstlisting}
\fi

\documentclass{article} 
\usepackage{amsthm} % writing proofs
\usepackage{amsfonts} % for blackboard bold charactes like Z R
\usepackage{amssymb} % came with intermediate value thrm
\usepackage[fleqn]{amsmath} % for arguments vertically underneath like lim, piecwise functions, equality arrays
\setlength{\mathindent}{0in} %neutralizes equation indentation
\usepackage{listings} % more literal, makes hw easier
\setlength{\parindent}{0in} %neutralizes annoying paragraph indent
\usepackage{mathtools} % for floor and ceiling functions / macros
\usepackage{nccmath} % for centering equations with ceqn
\usepackage{tkz-euclide} % geometry
\usetkzobj{all} % include all objects of tkz

\DeclarePairedDelimiter\ceil{\lceil}{\rceil}
\DeclarePairedDelimiter\floor{\lfloor}{\rfloor}

\newcommand{\f}{\frac}
\newcommand{\ra}{\rightarrow}
\newcommand{\rla}{\leftrightarrow}
\newcommand{\lra}{\leftrightarrow}
\newcommand{\bb}{\mathbb}

%\renewcommand{\qedsymbol}{$\blacksquare$}
\renewcommand{\qedsymbol}{$\dashv$}

\begin{document}

Jordan Winkler

stats

Mon Feb 18 12:02:09 EST 2019

hw6:
Sections 9.5-9.7 Exercise 9.14, 9.15, 9.20, 9.22, 9.23. 
Study the textbook Examples in Sections 9.5-9.7.  \\

9.14 The following measurements were recorded for the drying time, in hours, of a certain brand of latex paint: \\

3.4 2.5 4.8 2.9 3.6

2.8 3.3 5.6 3.7 2.8

4.4 4.0 5.2 3.0 4.8

Assuming that the measurements represent a random sample from a normal population, find a 95\% prediction interval for the drying time for the next trial of the paint. \\

\begin{eqnarray*}
\bar{x} & = & \f{\sum_{i=1}^n x_i }{n}  \\
& = &
(+ 3.4 2.5 4.8 2.9 3.6 2.8 3.3 5.6 3.7 2.8 4.4 4.0 5.2 3.0 4.8)/15  \\
& = & 56.8/15  \\
& \approx & 3.7866
\end{eqnarray*}

\begin{eqnarray*}
s & = & (\sum^n_{i = 1} \f{(x_i - \bar{x})^2}{n-1})^{1/2} \\
& = & (13.1973/14)^{1/2} \\
& = & 0.9709
\end{eqnarray*}

$\alpha = 0.05$

$df = n - 1 = 14$

$t_{.025,14} = 2.145$

$\bar{x} \pm t_{\alpha/2} \cdot s \cdot (1 + 1/n)^{1/2} $$= 3.7866 \pm 2.1508 $$\ra (1.6358, 5.9374)$ \\

9.15 Referring to Exercise 9.5, construct a 99\% prediction interval for the kilometers traveled annually by an automobile owner in Virginia. \\

from 9.5 

$n = 100$ 

$\bar{x} = 23500$

$s = 3900$ \\

$df = 99$ 

$\alpha = 0.01$

$t_{\alpha/2, 99} = 2.626$

\begin{eqnarray*}
\bar{x} \pm t_{\alpha/2,99} s (1 + 1/n)^{1/2} 
& = & 23500 \pm 2.626 \cdot 3900 \cdot (101/100) \\
& \approx & 23500 \pm 10292.4796 \\
& \ra & (13207.5204, 33792.4796) 
\end{eqnarray*}

9.20  Consider the situation of Exercise 9.11. Estimation of the mean diameter, while important, is not nearly as important as trying to pin down the location of the majority of the distribution of diameters. Find the 95\% tolerance limits that contain 95\% of the diameters. \\

from 9.11

$\bar{x} = \sum x_i /n \approx 1.0055 $ 

% (/ (+ 1.01 0.97 1.03 1.04 0.99 0.98 0.99 1.01 1.03) 9)

$s = (\f{\sum_{i=1}^n(x_i - \bar{x})^2}{n-1})^{1/2} = 0.0246$ 

$n = 9$  \\


$\alpha = 0.05 $

$\gamma = 0.05 $

$k = 3.532$ (* from table *) \\

% $k = (\f{(n-1)(1+1/n)z^2_{\alpha/2}}{\chi^2_\gamma (n-1)})^{1/2}$

% $= (\f{(8)(1+1/9)norm.cdf(0.025)^2}{chi2.ppf(0.05,8)})^{1/2}$

$\bar{x} \pm k s = 1.0056 \pm 3.532 \cdot 0.0246$ \\

$(0.9187,1.0924)$ \\


9.22  A type of thread is being studied for its tensile strength properties. Fifty pieces were tested under similar conditions, and the results showed an average tensile strength of 78.3 kilograms and a standard deviation of 5.6 kilograms. Assuming a normal distribution of tensile strengths, give a lower 95\% prediction limiton a single observed tensile strength value. In addition, give a lower 95\% tolerance limit that is exceeded by 99\% of the tensile strength values. \\

$df = n - 1 = 49$ \\

\begin{eqnarray*}
low_{.95} & = & \bar{x} - t_{\alpha,df}(s) (1 + 1/n)^{1/2}  \\
          & = & 78.3 - (1.677)(5.6)(1+1/50)^{1/2}  \\
          & = & 68.8153 \\
          & \ra & (68.8153, \infty)
\end{eqnarray*}


$\alpha = 0.05$

$\gamma = 0.01$

$k = 2.269$ \\

\begin{eqnarray*}
low_{.99} & =&  \bar{x} - ks \\
          & = & 78.3 - 2.269 * 5.6 \\
          & = & 65.5936 \\
          & \ra & (65.5936, \infty)
\end{eqnarray*}

9.23 Refer to Exercise 9.22. Why are the quantities requested in the exercise likely to be more important to the manufacturer of the thread than, say, a confidence interval on the mean tensile strength? \\

The primary information how much stress the metal threads can take. The customer could always use less force on the string. Extra info on the probability of this lower bound being correct for a particular string may also be helpful, but giving back an interval would sound silly, given that there is no upper bound.


\end{document}