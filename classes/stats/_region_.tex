\message{ !name(hw8.tex)}\iffalse
// c precompiler stuff
#define cnote \
\begin{lstlisting}[ \
  mathescape, \
  columns=fullflexible, \
  basicstyle=\fontfamily{lmvtt}\selectfont, \
] 

#define endcnote \end{lstlisting}
\fi

\documentclass{article} 
\usepackage{amsthm} % writing proofs
\usepackage{amsfonts} % for blackboard bold charactes like Z R
\usepackage{amssymb} % came with intermediate value thrm
\usepackage[fleqn]{amsmath} % for arguments vertically underneath like lim, piecwise functions, equality arrays
\setlength{\mathindent}{0in} %neutralizes equation indentation
\usepackage{listings} % more literal, makes hw easier
\setlength{\parindent}{0in} %neutralizes annoying paragraph indent
\usepackage{mathtools} % for floor and ceiling functions / macros
\usepackage{nccmath} % for centering equations with ceqn
%\usepackage{tkz-euclide} % geometry
%\usetkzobj{all} % include all objects of tkz

\DeclarePairedDelimiter\ceil{\lceil}{\rceil}
\DeclarePairedDelimiter\floor{\lfloor}{\rfloor}

\newcommand{\f}{\frac}
\newcommand{\ra}{\rightarrow}
\newcommand{\rla}{\leftrightarrow}
\newcommand{\lra}{\leftrightarrow}
\newcommand{\bb}{\mathbb}

%\renewcommand{\qedsymbol}{$\blacksquare$}
\renewcommand{\qedsymbol}{$\dashv$}

\begin{document}

\message{ !name(hw8.tex) !offset(-3) }


Jordan Winkler

stats

Mon Mar 18 10:26:41 EDT 2019

Homework 8:
Sections 9.12-9.13 Exercise 9.72, 9.74, 9.78, 9.80. 
Study the textbook Examples in Sections 9.12-9.13.

9.72 A random sample of 20 students yielded a mean of $\bar{x} = 72$ and a variance of $s^2 = 16$ for scores on a college placement test in mathematics. Assuming the scores to be normally distributed, construct a 98\% confidence interval for $\sigma^2$ . \\

$n = 20$

$\bar{x} = 72$

$s^2 = 16$

$a = 0.02$ \\

$(n-1)S^2/\chi^2_{\alpha/2} < \sigma^2 < (n-1)S^2/\chi^2_{1 - \alpha/2}$ \\

$(20-1)*16/36.191 < \sigma^2 < (20-1)*16/7.633$ \\

$8.3999 < \sigma^2 < 39.8271$ \\

9.74 Construct a 99\% confidence interval for $\sigma^2$ in Exercise 9.11 on page 283. \\

$\alpha = 0.01$ 

$n = 9$ 

$s = (1/(n-1) \sum^n_{i=1} (x_i - \bar{x})^2)^{1/2}$$= 0.0245$

summon formula:
$(n-1)S^2/\chi^2_{\alpha/2} < \sigma^2 < (n-1)S^2/\chi^2_{1 - \alpha/2}$ \\

compute: $0.0002 < \sigma^2 < 0.0036$ \\


9.78 Construct a 90\% confidence interval for $\sigma^2_1 /\sigma^2_2$ in Exercise 9.43 on page 295. Were we justified in assuming that $\sigma^2_1 = \sigma^2_2 $ when we constructed the confidence interval for $\mu_1 - \mu_2$ ? \\

$n_1 = 12$

$n_2 = 12$

$\bar{x_1} = 36300$

$\bar{x_2} = 38100$

$s_1 = 5000$

$s_2 = 6100$ \\

$\f{s_1^2}{s_2^2 * f_{\alpha/2}(v_1,v_2)} < 
\f{\sigma_1^2}{\sigma_2^2} 
< \f{s_1^2 * f_{\alpha/2}(v_1,v_2)}{s^2_2}$ \\

$\f{5000^2}{6100^2 * 2.82} < 
\f{\sigma_1^2}{\sigma_2^2} 
<\f{5000^2* 2.82}{6100^2 }$ \\

$ 0.2382 < \f{\sigma_1^2}{\sigma_2^2} < 1.8946 $ \\

So it is possible that $\sigma^2_1 = \sigma_2^2$ \\

9.80 Construct a 95\% confidence interval for $\sigma_A^2 / \sigma_B^2$ in Exercise 9.49 on page 295. Should the equal-variance assumption be used? \\

$n_1 = 15$

$\bar{x_A} = \sum(x_A)/n_A = 57.3/15 = 3.82$ \\

$s^2_A = \f{\sum(x_A-\bar{x_A})^2}/n_A-1 $$= 0.6074$ \\


$\bar{x_B} = \sum(x_B)/n_B = 4.94$

$s^2_B = \f{\sum(x_B-\bar{x_B})^2}/n_B-1 = 0.5682$ \\


$\f{s_A^2}{s_B^2 * f_{\alpha/2}(v_A,v_B)} < 
\f{\sigma_A^2}{\sigma_B^2} 
< \f{s_A^2 * f_{\alpha/2}(v_A,v_B)}{s^2_B}$ \\

$1.0688/2.98 < 
\f{\sigma_A^2}{\sigma_B^2}  
< 1.0688 * 2.98$ \\


$0.3586 < 
\f{\sigma_A^2}{\sigma_B^2}  
< 3.1850$ \\

So it is possible that $\sigma_A^2 = \sigma_B^2$


\message{ !name(hw8.tex) !offset(-5) }

\end{document}