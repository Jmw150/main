\iffalse
// c precompiler stuff
#define cnote \
\begin{lstlisting}[ \
  mathescape, \
  columns=fullflexible, \
  basicstyle=\fontfamily{lmvtt}\selectfont, \
] 

#define endcnote \end{lstlisting}
\fi

\documentclass{article} 
\usepackage{amsthm} % writing proofs
\usepackage{amsfonts} % for blackboard bold charactes like Z R
\usepackage{amssymb} % came with intermediate value thrm
\usepackage[fleqn]{amsmath} % for arguments vertically underneath like lim, piecwise functions, equality arrays
\setlength{\mathindent}{0in} %neutralizes equation indentation
\usepackage{listings} % more literal, makes hw easier
\setlength{\parindent}{0in} %neutralizes annoying paragraph indent
\usepackage{mathtools} % for floor and ceiling functions / macros
\usepackage{nccmath} % for centering equations with ceqn
\usepackage{tkz-euclide} % geometry
\usetkzobj{all} % include all objects of tkz

\DeclarePairedDelimiter\ceil{\lceil}{\rceil}
\DeclarePairedDelimiter\floor{\lfloor}{\rfloor}

\newcommand{\f}{\frac}
\newcommand{\ra}{\rightarrow}
\newcommand{\rla}{\leftrightarrow}
\newcommand{\lra}{\leftrightarrow}
\newcommand{\bb}{\mathbb}

%\renewcommand{\qedsymbol}{$\blacksquare$}
\renewcommand{\qedsymbol}{$\dashv$}

\begin{document}

Jordan Winkler

stats

hw2

Tue Jan 22 15:43:55 EST 2019 \\

Pages 230-232. Exercises. 8.1, 8.8, 8.13, 8.15.

Using the data in Exercise 1.25 on Page 32 

(a) to find sample mean, sample median, and the 10\% trimmed sample mean; 

(b) to construct a relative frequency histogram using class boundaries 0, 20, 40, 60, 80, 100;

(c) to find Q1, Q2, Q3, and IQR;

(d) to construct a box-and-whisker plot;

Study the textbook Examples in Sections 8.1-8.2. \\


8.1
Define suitable populations from which the following samples are selected: \\

(a) Persons in 200 homes in the city of Richmond are called on the phone and asked to name the candidate they favor for election to the school board. \\

The responses of the people in the 200 homes in the city of Richmond. \\

(b) A coin is tossed 100 times and 34 tails are recorded.

The results of the 100 coin tosses. \\

(c) Two hundred pairs of a new type of tennis shoe were tested on the professional tour and, on average, lasted 4 months. \\

The results of the test on the two hundred pairs of tennes shoes. \\

(d) On five different occasions it took a lawyer 21, 26, 24, 22, and 21 minutes to drive from her suburban home to her midtown office. \\

The time intervals just mentioned. \\[.5 cm]




8.8 According to ecology writer Jacqueline Killeen, phosphates contained in household detergents pass right through our sewer systems, causing lakes to turn into swamps that eventually dry up into deserts. The following data show the amount of phosphates per load of laundry, in grams, for a random sample of various types of detergents used according to the prescribed directions: \\

A n P Blue Sail, 48

Dash, 47

Concentrated All, 42

Cold Water All, 42

Breeze, 41

Oxydol, 34

Ajax, 31

Sears, 30

Fab, 29

Cold Power, 29

Bold, 29

Rinso, 26 \\

For the given phosphate data, find

(a) the mean; 35.6666

(b) the median; 32.5

(c) the mode. 29 \\

[attached] \\[0.5cm]

8.13 The grade-point averages of 20 college seniors selected at random from a graduating class are as follows: \\

(define gpa 
(list 3.2 1.9 2.7 2.4 2.8 2.9 3.8 3 2.5 3.3 1.8 2.5 3.7 2.8 2 3.2 2.3 2.1 2.5 1.9))

give standard deviation. 0.5851 \\[.5cm]

8.15. Verify that the variance of the sample 4, 9, 3, 6, 4, and 7 is 5.1, and using this fact, along with the results of Exercise 8.14, find \\

(a) the variance of the sample 12, 27, 9, 18, 12, and 21; \\

(b) the variance of the sample 9, 14, 8, 11, 9, and 12. \\


(extra) Using the data in Exercise 1.25 on Page 32  \\

72.2	31.9	26.5	29.1	27.3	8.6	22.3	26.5

20.4	12.8	25.1	19.2	24.1	58.2	68.1	89.2

55.1	9.4	14.5	13.9	20.7	17.9	8.5	55.4

38.1	54.2	21.5	26.2	59.1	43.3		 \\

(a) to find sample mean, sample median, and the 10\% trimmed sample mean; \\


(b) to construct a relative frequency histogram using class boundaries 0, 20, 40, 60, 80, 100; \\

(c) to find Q1, Q2, Q3, and IQR; \\

(d) to construct a box-and-whisker plot; \\


\end{document}