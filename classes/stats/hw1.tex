\iffalse
// c precompiler stuff
#define cnote \
\begin{lstlisting}[ \
  mathescape, \
  columns=fullflexible, \
  basicstyle=\fontfamily{lmvtt}\selectfont, \
] 

#define endcnote \end{lstlisting}
\fi

\documentclass{article} 
\usepackage{amsthm} % writing proofs
\usepackage{amsfonts} % for blackboard bold charactes like Z R
\usepackage{amssymb} % came with intermediate value thrm
\usepackage[fleqn]{amsmath} % for arguments vertically underneath like lim, piecwise functions, equality arrays
\setlength{\mathindent}{0in} %neutralizes equation indentation
\usepackage{listings} % more literal, makes hw easier
\setlength{\parindent}{0in} %neutralizes annoying paragraph indent
\usepackage{mathtools} % for floor and ceiling functions / macros
\usepackage{nccmath} % for centering equations with ceqn
\usepackage{tkz-euclide} % geometry
\usetkzobj{all} % include all objects of tkz

\DeclarePairedDelimiter\ceil{\lceil}{\rceil}
\DeclarePairedDelimiter\floor{\lfloor}{\rfloor}

\newcommand{\f}{\frac}
\newcommand{\ra}{\rightarrow}
\newcommand{\rla}{\leftrightarrow}
\newcommand{\lra}{\leftrightarrow}
\newcommand{\bb}{\mathbb}

%\renewcommand{\qedsymbol}{$\blacksquare$}
\renewcommand{\qedsymbol}{$\dashv$}

\begin{document}

Jordan Winkler

stats

Mon Jan  7 16:16:14 EST 2019

hw 1: pg 193 6.24, 6.25, 6.31, 6.33, 6.36. 

pg 206, 6.40, 6.43, 6.45, 6.46, 6.55, 6.58, 6.59, 6.81. 

Study the textbook Examples 6.15-6.16 in Section 6.5.

Study the textbook Examples 6.17-6.21 in Sections 6.6-6.7. \\

24. A coin is tossed 400 times. Use the normal curve approximation to find the probability of obtaining \\

a) between 185 and 210 heads inclusive; \\

$n = 400$

$p = 0.5$

$\mu = np = 200$

$\sigma = (np(1-p))^{1/2} = 10 $

$z_a = \f{185 - 200}{10} = -1.5$

$z_b = \f{210 - 200}{10} = 1$ \\


\begin{eqnarray*}
P(z_a < Z < z_b) & = & P(-1.5 < Z < 1) \\ 
  & =&  cdf(1) - cdf(-1.5)  \\
& = & 0.7745375447996848 
\end{eqnarray*}

% Or $cdf(1.05) - cdf(-1.55) = 0.7925701856220451$ to be more inclusive.

b) exactly 205 heads; \\

Well this model gives a 0 probability for exact values, because it is a real number distribution. But the standard method is to give the probability of ``about'' 205 heads or 204.5-205.5. \\

$z_a = (204.5 - 200)/10 = 0.45$

$z_b = (205.5 - 200)/10 = 0.55$

$P(z_a < Z < z_b) = cdf(.55) - cdf(.45) = 0.035195533499573606$ \\

c) fewer than 176 or more than 227 heads. \\

$z_a = (176 - 200)/10 = -2.4$

$z_b = (227 - 200)/10 = 2.7$

\begin{eqnarray*}
P(Z > z_a) + P(z_b < Z) & = & cdf(-2.4) + (1 - cdf(2.7)) \\
                        & \approx & 0.01166450972763675 
\end{eqnarray*}

25. A process for manufacturing an electronic com-
ponent yields items of which 1\% are defective. A qual-
ity control plan is to select 100 items from the process,
and if none are defective, the process continues. Use
the normal approximation to the binomial to find \\

(a) the probability that the process continues given the
sampling plan described; \\

$p = 0.01$

$n = 100$

$\mu = np = 1$

$\sigma = (npq)^{1/2} = 
0.99498743710662
$ \\

$cc = $ continuity correction \\

$z = \f{x+ cc - \mu}{\sigma} = \f{0 + .5 - 1}{
0.99498743710662
} \approx 
-0.502518907629606
$ 

$P(Z < z) \approx cdf(-0.502518907629606) 
\approx 
0.30765127784525537
$ \\

(b) the probability that the process continues even if the process has gone bad (i.e., if the frequency of defective components has shifted to 5.0\% defective). \\

\begin{eqnarray*}
P(Z \leq \f{x + cc - np}{(npq)^{1/2}})    & \approx & 
P(Z \leq \f{0 + .5 - 100*.05}{(100*.05*.95)^{1/2}}) \\ & \approx & 
P(Z \leq -2.0647416048350555) \\ & \approx &
cdf(-2.0647416048350555) \\ & \approx & 0.019473727871012713
\end{eqnarray*}

31. One-sixth of the male freshmen entering a large
state school are out-of-state students. If the students
are assigned at random to dormitories, 180 to a build-
ing, what is the probability that in a given dormitory
at least one-fifth of the students are from out of state? \\

$X := Binomial(180,1/6)$ \\

$\mu = np = 180(1/6) = 30$

$\sigma = (npq)^{1/2} = (180 * 1/6 * 5/6)^{1/2} = 5$

\begin{eqnarray*}
P(X \geq 180 * 1/5) & = & P(X \geq 36)  \\
                    & \approx & P(X \geq 36 - 0.5) \\
 & \approx & P(\f{X - \mu}{\sigma} \geq \f{35.5 - 30}{5}) \\
& = & P(Z \geq 1.1) \\
& = & 1- cdf(1.10) \\
& \approx & 0.13566606094638267
\end{eqnarray*}

33. Statistics released by the National Highway
Traffic Safety Administration and the National Safety
Council show that on an average weekend night, 1 out
of every 10 drivers on the road is drunk. If 400 drivers
are randomly checked next Saturday night, what is the
probability that the number of drunk drivers will be \\

(a) less than 32? \\

$p = 0.1$

$n = 400$

$\mu = np = 400 * .1 = 40$

$\sigma = (npq)^{1/2} = 6$

\begin{eqnarray*}
P(X \leq 32) & = & P(\f{X - 0.5 - \mu}{\sigma} \leq \f{32 - .5 -40 }{6}) \\
& \approx & P(Z < -1.4166666666666667) \\
& \approx & cdf(-1.42) \\
& \approx & 0.07780384052654638
\end{eqnarray*}

(b) more than 49? \\

\begin{eqnarray*}
P(X > 49) & = & P(\f{(X + 0.5)-\mu}{\sigma} > \f{49+.5-40}{6}) \\
  & \approx & P(Z > 1.583333333333333) \\
  & = & 1 - cdf(1.583333333333333) \\
& \approx & 0.056672754609762954 \\
\end{eqnarray*}

(c) at least 35 but less than 47? \\

\begin{eqnarray*}
P(35 \leq X < 47) & \approx & P(\f{34.5 - 40}{6} < \f{X - \mu}{\sigma} < \f{46.5 - 40}{6}) \\
& \approx & P(-0.92 < Z < 1.08) \\
& = & cdf(1.08) - cdf(-0.92) \\
& \approx & 0.6811425302968593
\end{eqnarray*}

36. A common practice of airline companies is to sell more tickets for a particular flight than there are seats on the plane, because customers who buy tickets do not always show up for the flight. Suppose that the percentage of no-shows at flight time is 2\%. For a particular flight with 197 seats, a total of 200 tick-ets were sold. What is the probability that the airline
overbooked this flight? \\

$n = 200$

$p = 0.98$

$\mu = np = 200 * .98 = 196$

$\sigma = (npq)^{1/2} = (200 * .98 * .02)^{1/2} = 1.9798989873223332$

\begin{eqnarray*}
P(X > 197) & \approx &  P(Z > \f{197.5 - 196}{1.98}) \\
& \approx & 1 - cdf(0.7575757575757576) \\
& \approx & 0.22435249875681773
\end{eqnarray*}


40. In a certain city, the daily consumption of water
(in millions of liters) follows approximately a gamma distribution with $\alpha = 2$ and $\beta = 3$. If the daily capacity of that city is 9 million liters of water, what is the probability that on any given day the water supply is inadequate? \\

$gammapdf(x,\alpha, \beta) = \f{1}{\beta^2 \Gamma(\alpha)} x^{\alpha - 1} e^{-x/\beta} , x > 0, $ else $ 0$

\begin{eqnarray*}
P(X > 9) & = & 1 -  P(X \leq 9) \\
& = & 1- \int^9_0 gammapdf(x,2,3) dx \\
& = & 1- \int^9_0 1/9 * x e^{-x/3} dx \\
& = & 1- \f{1}{9} (-3xe^{-x/3} - 9e^{-x/3})|^9_0 \\
& = & 1- (1 - 4e^{-3}) \\
& = & 4e^{-3}  \\
& \approx & 0.19914827347145578
\end{eqnarray*}


43. (a) Find the mean and variance of the daily wa-
ter consumption in Exercise 6.40. \\

$E(X) = \mu = \alpha \beta = 2 * 3 = 6$

$V(X) = \sigma^2 = \alpha \beta^2 = 2 * 3^ 2 = 18$ \\

(b) According to Chebyshev’s theorem, there is a prob-
ability of at least 3/4 that the water consumption
on any given day will fall within what interval? \\

$x_a = \mu - 2\sigma = 6 - 2(18)^{1/2} = -2.4852813742385695 $

$x_b = \mu + 2\sigma = 6 + 2(18)^{1/2} = 14.48528137423857 $ \\

Negative water consumption sounds ridiculous so lets say the interval is $[0,14.48528137423857 ]$. \\

45. The length of time for one individual to be
served at a cafeteria is a random variable having an ex-
ponential distribution with a mean of 4 minutes. What
is the probability that a person is served in less than 3
minutes on at least 4 of the next 6 days? \\

$exppdf(x,4) = \f{1}{4} e^{-x/4}$ if $x \geq 0$, else $0$.

\begin{eqnarray*}
P(X < 3) & = & \int_0^3 \f{1}{4} e^{-x/4} dx \\
  & = & -e^{-x/4} |^3_0 \\
  & = & -(e^{-3/4} - 1) \\
  & = & 1 - e^{-3/4} \\
  & = & 0.5276334472589853
\end{eqnarray*}

\begin{eqnarray*}
P(X \geq 4) & = & \sum^6_{x=4} {6 \choose x}  p^x q^{6-x} \\
& = & P(X = 4) + P(X = 5) + P(X = 6) \\
& \approx & 0.3968846998826859 
\end{eqnarray*}

46. The life, in years, of a certain type of electrical
switch has an exponential distribution with an average
life $\beta$ = 2. If 100 of these switches are installed in dif-
ferent systems, what is the probability that at most 30
fail during the first year? \\

$X := Exp(\beta = 2)$  \\

$p = $$P(X \geq 1) = 1 - e^{-1/2} \approx 0.3934693402873666$ \\

$n = 100$


\begin{eqnarray*}
P(X \leq 30) & \approx & P(X \leq 30.5) \\
             & = & P(Z \leq \f{30.5 - np}{(npq)^{1/2}}) \\
             & = & P(Z \leq \f{30.5 - 100* 0.393469}{(100 * 0.3934693 * (1- 0.393469340))^{1/2}}) \\
& \approx & P(Z \leq -1.8109687685853402) \\
& = & cdf(-1.81) \\
& \approx & 0.03514789358403879
\end{eqnarray*}

%(/ (+ 30.5 (* -0.3934693402873666 100)) (expt (* 100 0.3934693402873666 (- 1 0.3934693402873666)) .5))

55. Computer response time is an important application of the gamma and exponential distributions. Suppose that a study of a certain computer system reveals that the response time, in seconds, has an exponential distribution with a mean of 3 seconds. \\

(a) What is the probability that response time exceeds 5 seconds? \\

\begin{eqnarray*}
P(X > 5) & = & 1 - P(X \leq 5) \\
  & = & 1 - \f{1}{3} \int^5_0 e^{-x/3} dx \\
  & = & 1 - (1 - e^{-5/3}) \\
  & = & e^{-5/3} \\
  & \approx & 0.18887560283756183 
\end{eqnarray*}

(b) What is the probability that response time exceeds 10 seconds? \\

\begin{eqnarray*}
P(x > 10) & = & 1 - \f{1}{3} \int^{10}_0 e^{-x/3} dx \\
  & = & e^{-10/3} \\
  & \approx & 0.035673993347252395
\end{eqnarray*}

58. The number of automobiles that arrive at a certain intersection per minute has a Poisson distribution with a mean of 5. Interest centers around the time that elapses before 10 automobiles appear at the intersection. \\

a) What is the probability that more than 10 automobiles appear at the intersection during any given minute of time? \\

$X := Poisson(\gamma = 5)$

\begin{eqnarray*}
P(X > 10) & = &  1 - P(X \leq 10) \\
& = & 1 - \sum_x^{10} e^{-5}5^x/x! \\
& \approx & 1 -  0.9863047314016171 \\
& = & 0.013695268598382881
\end{eqnarray*}

(b) What is the probability that more than 2 minutes
elapse before 10 cars arrive? \\

$gammapdf(x,\alpha,\beta) = \f{y^{\alpha-1}e^{-y/\beta}}{\Gamma(\alpha) \beta^\alpha}$

$\alpha = 10$

$\beta = 2/10 = 1/5$

\begin{eqnarray*}
P(X > 2) & = & P(Y > 10)  \\
            & = & 1- \int^{10}_0 y^9e^{-y}/\Gamma(10) dy \\
            & = & 1- \f{1}{9!}\int^{10}_0 y^9e^{-y} dy \\
            & = & 1 -(-\f{3660215680}{e^{10}} + 362880) \\
            & \approx & 1- 0.5421 \\
            & = & 0.4579
\end{eqnarray*}

59. Consider the information in Exercise 6.58. \\

(a) What is the probability that more than 1 minute
elapses between arrivals? \\

\begin{eqnarray*}
P(X > 1) & = & 1 - P(X \leq 1) \\
         & = & 1 - \int^1_0 5e^{-5x} dx \\
         & = & 1 - 10(e^{-5x}/(-5)) |^1_0 \\
         & = & e^{-5} \\
         & \approx & 0.006737946999085467
\end{eqnarray*}

(b) What is the mean number of minutes that elapse
between arrivals? \\

$\beta = 1/\gamma = 1/5$ \\

81. The length of time between breakdowns of an es-
sential piece of equipment is important in the decision
of the use of auxiliary equipment. An engineer thinks
that the best model for time between breakdowns of a
generator is the exponential distribution with a mean
of 15 days. \\

(a) If the generator has just broken down, what is the
probability that it will break down in the next 21
days? \\

\begin{eqnarray*}
P(X \leq 21) &=& \f{1}{15} \int^\infty _0  e^{-x/15} dx \\
& = & 1 - e^{-21/15}  \\
& \approx & 0.7534030360583935
\end{eqnarray*}

(b) What is the probability that the generator will op-
erate for 30 days without a breakdown? \\

\begin{eqnarray*}
P(X > 30) &=& 1 - \f{1}{15} \int^{30} _0  e^{-x/15} dx \\
& = & e^{-30/15}  \\
& \approx & 0.1353352832366127
\end{eqnarray*}

\end{document}