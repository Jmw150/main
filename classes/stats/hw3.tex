\iffalse
// c precompiler stuff
#define cnote \
\begin{lstlisting}[ \
  mathescape, \
  columns=fullflexible, \
  basicstyle=\fontfamily{lmvtt}\selectfont, \
] 

#define endcnote \end{lstlisting}
\fi

\documentclass{article} 
\usepackage{amsthm} % writing proofs
\usepackage{amsfonts} % for blackboard bold charactes like Z R
\usepackage{amssymb} % came with intermediate value thrm
\usepackage[fleqn]{amsmath} % for arguments vertically underneath like lim, piecwise functions, equality arrays
\setlength{\mathindent}{0in} %neutralizes equation indentation
\usepackage{listings} % more literal, makes hw easier
\setlength{\parindent}{0in} %neutralizes annoying paragraph indent
\usepackage{mathtools} % for floor and ceiling functions / macros
\usepackage{nccmath} % for centering equations with ceqn
\usepackage{tkz-euclide} % geometry
\usetkzobj{all} % include all objects of tkz

\DeclarePairedDelimiter\ceil{\lceil}{\rceil}
\DeclarePairedDelimiter\floor{\lfloor}{\rfloor}

\newcommand{\f}{\frac}
\newcommand{\ra}{\rightarrow}
\newcommand{\rla}{\leftrightarrow}
\newcommand{\lra}{\leftrightarrow}
\newcommand{\bb}{\mathbb}

%\renewcommand{\qedsymbol}{$\blacksquare$}
\renewcommand{\qedsymbol}{$\dashv$}

\begin{document}

Jordan Winkler

stats

Wed Jan 30 13:22:23 EST 2019

hw3: 

Sections 8.3-8.4. Exercises 8.18, 8.19, 8.21, 8.26, 8.32, 8.34
Study the textbook Examples in Sections 8.3-8.4.  \\

8.18 If the standard deviation of the mean for the
sampling distribution of random samples of size 36
from a large or infinite population is 2, how large must
the sample size become if the standard deviation is to
be reduced to 1.2? \\

$n = 36$

$\sigma_{\bar{x}} = 2$

$\sigma_x = n^{1/2}\sigma_{\bar{x}} = 6 \cdot 2 = 12$  \\

note: $ \sigma_{\bar{x}} = \f{\sigma_x}{n^{1/2}}$ so 

$n_{\sigma = 1.2} = (\f{12}{1.2})^2 = 100$ \\


8.19 A certain type of thread is manufactured with a
mean tensile strength of 78.3 kilograms and a standard
deviation of 5.6 kilograms. How is the variance of the
sample mean changed when the sample size is

(a) increased from 64 to 196? \\

$\sigma_{n=64}^2 = \sigma^2/n = 5.6^2/64 \approx 0.4899$ \\

$\sigma_{n=196}^2 = \sigma^2/n = 5.6^2/196 \approx 0.1599$ \\

So a sample variance reduction of $\sigma^2_{n=64} - \sigma^2_{n=196} = 0.33$

(b) decreased from 784 to 49? \\


$\sigma_{n=784}^2 = \sigma^2/n = 5.6^2/784 \approx 
0.0399
$ \\

$\sigma_{n=49}^2 = \sigma^2/n = 5.6^2/49 \approx 
0.6399
$ \\

$\sigma^2_{n=64} - \sigma^2_{n=784} = 0.6399 - 0.0399 = 0.6$ \\


8.21 A soft-drink machine is regulated so that the
amount of drink dispensed averages 240 milliliters with
a standard deviation of 15 milliliters. Periodically, the
machine is checked by taking a sample of 40 drinks
and computing the average content. If the mean of the
40 drinks is a value within the interval $\mu_{\bar{X}} \pm 2\simga_{\bar{X}}$, the
machine is thought to be operating satisfactorily; otherwise, adjustments are made. In Section 8.3, the company official found the mean of 40 drinks to be $\bar{x} = 236$
milliliters and concluded that the machine needed no
adjustment. Was this a reasonable decision? \\

$n = 40$

$\bar{x} = 240$

$\sigma_x = 15$ 

$\sigma_{\bar{x}} = \sigma_x/n^{1/2} = 15/40^{1/2} \approx 2.3717$ \\

$(\bar{x} - 2\sigma_{\bar{x}} , \bar{x} + 2\sigma_{\bar{x}}) = (
235.2566, 244.7434)$ \\

8.26 The amount of time that a drive-through bank
teller spends on a customer is a random variable with
a mean μ = 3.2 minutes and a standard deviation
σ = 1.6 minutes. If a random sample of 64 customers
is observed, find the probability that their mean time
at the teller’s window is \\

$\mu = 3.2$

$\sigma = 1.6$

$n = 64$

$\sigma_{\bar{x}} = \sigma/n^{1/2} = 1.6/8 = 0.2$ \\

(a) at most 2.7 minutes; \\

$P(\bar{x} < 2.7) = P(Z < \f{2.7-3.2}{0.2}) = P(Z < -2.5) \approx 0.0062$ \\

(b) more than 3.5 minutes; \\

$P(\bar{x} > 3.5) = P(Z > \f{3.5 - 3.2}{0.2}) = P(Z > 1.5) \approx 0.0668$ \\

(c) at least 3.2 minutes but less than 3.4 minutes. \\

\begin{eqnarray*}
P(3.2 \leq \bar{x} \leq 3.4) & = & P(\f{3.2 - 3.2}{0.2} \leq Z \leq \f{3.4 - 3.2}{0.2})  \\
& = &  P(0 < Z < 1) \\
& = &  P(1 > Z) - P(Z < 0) \\
& \approx & 0.3413
\end{eqnarray*}


8.32 Two different box-filling machines are used to fill
cereal boxes on an assembly line. The critical measure-
ment influenced by these machines is the weight of the
product in the boxes. Engineers are quite certain that
the variance of the weight of product is σ 2 = 1 ounce.
Experiments are conducted using both machines with
sample sizes of 36 each. The sample averages for ma-
chines $A$ and $B$ are $\bar{x}_A = 4.5$ ounces and $\bar{x}_B = 4.7$ ounces. Engineers are surprised that the two sample
averages for the filling machines are so different. \\

$n_A = n_B = 36$

$\sigma^2_A = \sigma^2_B = 1$

$\sigma^2_{\bar{x}_A - \bar{x}_B} = 
\f{\sigma^2_A}{n_A} 
+ \f{\sigma^2_B}{n_B}  
= 2/36 = 1/18$ \\

$\sigma_{\bar{x}_A - \bar{x}_B} = (1/18)^{1/2} \approx 0.2357$ \\

(a) Use the Central Limit Theorem to determine $P(\bar{X}_B - \bar{X}_A \geq 0.2)$
under the condition that $\mu_A = \mu_B$. \\

\begin{eqnarray*}
P(\bar{X}_B - \bar{X}_A \geq 0.2) 
& = & P(\f{\bar{x}_A - \bar{x}_B) - (\mu_A - \mu_B)}{(
\f{\sigma^2_A}{n_A} + \f{\sigma^2_B}{n_B}  )^{1/2}} \geq \f{0.2}{.2357}) \\
& \approx & P(Z \leq 0.8485) \\
& = & 1- P(Z \geq 0.8485) \\
& \approx & 0.1977
\end{eqnarray*}


(b) Do the aforementioned experiments seem to, in any
way, strongly support a conjecture that the popu-
lation means for the two machines are different?
Explain using your answer in (a). \\

The probability of that much difference or more is about 20 percent. So there is an 80 percent chance that this could happen. That seems pretty unreliable compared to the usual 95 percent confidence. So I am going to say the conjecture was false. \\ 

8.34 Two alloys A and B are being used to manufacture a certain steel product. An experiment needs to
be designed to compare the two in terms of maximum
load capacity in tons (the maximum weight that can
be tolerated without breaking). It is known that the
two standard deviations in load capacity are equal at
5 tons each. An experiment is conducted in which 30
specimens of each alloy (A and B) are tested and the
results recorded as follows: \\

$\bar{x}_A = 49.5$

$\bar{x}_B = 45.5$

$\bar{x}_A - \bar{x}_B = 4$ \\

The manufacturers of alloy A are convinced that this
evidence shows conclusively that $\mu_A > \mu_B$ and strongly
supports the claim that their alloy is superior. Man-
ufacturers of alloy B claim that the experiment could
easily have given $\bar{x}_A - \bar{x}_B = 4$ even if the two population means are equal. In other words, “the results are
inconclusive!” \\


(a) Make an argument that manufacturers of alloy B
are wrong. Do it by computing $P(\bar{X}_A - \bar{X}_B > 4 | \mu_A = \mu_B)$ \\

$n_A = n_B = 30$

$\sigma_A = \sigma_B = 5$ 

$\bar{x}_A = 49.5$

$\bar{x}_B = 45.5$ \\

$\sigma^2_{\bar{x}_A - \bar{x}_B} = 2 \cdot \f{25}{30} = 1.6667$


$\sigma_{\bar{x}_A - \bar{x}_B} = 1.2910$ \\
\begin{eqnarray*}
P(\bar{X}_A - \bar{X}_B > 4 | \mu_A = \mu_B) & = & 
P(Z \leq \f{4-0}{1.2910}) \\
& \approx & P(Z \leq 3.0984) \\
& \approx & 0.0010 \\
\end{eqnarray*}

(b) Do you think these data strongly supports alloy A? \\

Yes. $P(\bar{X}_A - \bar{X}_B > 4 | \mu_A = \mu_B)$ is a nice low number. So the probability that our conjecture is correct is pretty high. 

\end{document}