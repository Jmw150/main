\iffalse
// c precompiler stuff
#define cnote \
\begin{lstlisting}[ \
  mathescape, \
  columns=fullflexible, \
  basicstyle=\fontfamily{lmvtt}\selectfont, \
] 

#define endcnote \end{lstlisting}
\fi

\documentclass{article} 
\usepackage{amsthm} % writing proofs
\usepackage{amsfonts} % for blackboard bold charactes like Z R
\usepackage{amssymb} % came with intermediate value thrm
\usepackage[fleqn]{amsmath} % for arguments vertically underneath like lim, piecwise functions, equality arrays
\setlength{\mathindent}{0in} %neutralizes equation indentation
\usepackage{listings} % more literal, makes hw easier
\setlength{\parindent}{0in} %neutralizes annoying paragraph indent
\usepackage{mathtools} % for floor and ceiling functions / macros
\usepackage{nccmath} % for centering equations with ceqn
\usepackage{tkz-euclide} % geometry
\usetkzobj{all} % include all objects of tkz

\DeclarePairedDelimiter\ceil{\lceil}{\rceil}
\DeclarePairedDelimiter\floor{\lfloor}{\rfloor}

\newcommand{\f}{\frac}
\newcommand{\ra}{\rightarrow}
\newcommand{\rla}{\leftrightarrow}
\newcommand{\lra}{\leftrightarrow}
\newcommand{\bb}{\mathbb}

%\renewcommand{\qedsymbol}{$\blacksquare$}
\renewcommand{\qedsymbol}{$\dashv$}

\begin{document}

Jordan Winkler

stats

Statistics is an application of probability theory. \\

It all starts with the concet of a random variable. \\

random-variable : { Things-that-could-happen } -> some-measureable-space, given integral = 1

definitions
Probability space is a measure space $(X,P,\mu)$ with the extra restriction $\mu(X) = 1$. $X$ is the set of possibilities. $P$ is a sigma algebra or 
measure space = $(\Omega, F, P)$

$\Omega$ = 'sample space' = elements of 
$F$ = 'event space' = sigma algebra 
$P$ = 'probability measure' = 





This is defined within 3 probability axioms.

axiom 1:
$\forall E \in F$ $P(E) \in \bb R$, $P(E) \geq 0$

The typical measure space is discrete. So combinatorics!







\end{document}